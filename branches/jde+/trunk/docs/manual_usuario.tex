\documentclass[a4paper,10pt]{article}
\usepackage[latin1]{inputenc}
\usepackage[spanish]{babel}

%opening
\title{Manual de usuario de JDE+\\Versi�n 0.0}
\author{David Lobato Bravo\\david.lobato@urjc.es}

\begin{document}

\maketitle

\begin{abstract}
Este documento describe como instalar y usar la plataforma de desarrollo JDE+. El documento esta orientado a los desarroladores de aplicaciones sobre JDE+, mostrandose las herramientas que ofrece y la forma de usarlas. Tambi�n se describe el API de programaci�n y se comentan algunos ejemplos.
\end{abstract}

\section{Introducci�n}
TODO

\section{Instalaci�n de JDE+}
Lo primero es obtener el paquete (FIXME: a�adir fuente). Una vez lo tengamos, lo descomprimimos con:
\begin{verbatim}
tar zxvf jde+'version'.tar.gz
\end{verbatim}

Al descomprimirlo, obtendremos un directorio con todo el software. El proyecto JDE+ esta gestionado con AutoTools, por lo que tan solo tendremos que configurarlo, compilarlo e instalarlo con los siguientes comandos:
\begin{verbatim}
./configure
make
make install
\end{verbatim}

El directorio de instalaci�n por defecto es \texttt{/usr/local/jde+-'version'}. Si queremos modificarlo, podemos usar el par�metro \texttt{--prefix} de configure con una nueva ruta de instalaci�n.

Por �ltimo definiremos las siguientes variables de entorno en nuestro \texttt{.bashrc}:
\begin{verbatim}
JDE_DIR=/usr/local/jde+-'version'
LD_LIBRARY_PATH=$JDE_DIR/lib:$LD_LIBRARY_PATH
PATH=$JDE_DIR/bin:$PATH

export JDE_DIR LD_LIBRARY_PATH PATH
\end{verbatim}

Con esto deberiamos poder ejecutar el comando \texttt{jde+}, que nos mostrar� la interfaz gr�fica de JDE+.

\end{document}
